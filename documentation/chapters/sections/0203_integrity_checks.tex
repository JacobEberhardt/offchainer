\section{Integrity Checks}
\sectionnames{Simon Fallnich}


	This section describes basic principles to verify the integrity of data from an untrusted source.

	% Hashing
	\subsection{Hashing}
	\label{subsec:hashing}

		A \emph{hash function} maps an input of arbitrary length to an ouput of fixed length, referred to as \emph{hash} \cite{menezes1996}.
		Although hash functions are used for a wide range of applications in the domain of computer science, the class of \emph{cryptographic hash functions} is primarily relevant for our application.
		A cryptographic hash function, or \emph{one-way function}, is a hash function which is infeasible to invert.
		This essential property lays the foundation for efficient integrity check mechanisms as the resulting hashes can be used to securely proof equality of the underlying data.

	% Merkle Tree
	\subsection{Merkle Tree}
	\label{subsec:merkle-tree}

		\begin{figure}
			\centering
				\begin{tikzpicture}[line cap=round, line join=round, yscale=0.8]
					% First layer
					\node [circle, draw=black, fill=white, minimum size = 1cm] (node-1-1) at (0, 0) {$H_0$};
					% Second layer
					\foreach \a [count=\x] in {1, 2}
						\node [circle, draw=black, fill=white, minimum size=1cm] (node-2-\x) at (-12 + 8 * \x, -2) {$H_\a$};
					% Third layer
					\foreach \a [count=\x] in {3, ..., 6}
					  \node [circle, draw=black, fill=white, minimum size=1cm] (node-3-\x) at (-10 + 4 * \x, -4) {$H_\a$};
					% Fourth layer
					\foreach \a [count=\x] in {7, 8}
					  \node [circle, draw=black, fill=white, minimum size=1cm] (node-4-\x) at (-9 + 2 * \x, -6) {$H_\a$};
					% First data layer
					\foreach \a [count=\x] in {0, ..., 2}
					  \node [draw=gray, text=gray, fill=white, minimum size=0.8cm] (data-1-\x) at (-6 + 4 * \x, -6) {$D_\a$};
					% Second data layer
					\foreach \a [count=\x] in {3, 4}
					  \node [draw=gray, text=gray, fill=white, minimum size=0.8cm] (data-2-\x) at (-9 + 2 * \x, -8) {$D_\a$};
					% First layer arrows
					\foreach \a in {1, 2}
					  \draw (node-1-1) -- (node-2-\a);
					% Second layer arrows
					\foreach \a in {1, 2}
					  \draw (node-2-1) -- (node-3-\a);
					\foreach \a in {3, 4}
					  \draw (node-2-2) -- (node-3-\a);
					% Third layer arrows
					\foreach \a in {1, 2}
					  \draw (node-3-1) -- (node-4-\a);
					\draw [draw=gray] (node-3-2) -- (data-1-1);
					\draw [draw=gray] (node-3-3) -- (data-1-2);
					\draw [draw=gray] (node-3-4) -- (data-1-3);
					\draw [draw=gray] (node-4-1) -- (data-2-1);
					\draw [draw=gray] (node-4-2) -- (data-2-2);
				\end{tikzpicture}
				\caption[Merkle Tree]{A binary Merkle tree with five leaf nodes. While $H_i$ denotes the hash which is stored for the node with index $i$, $D_j$ denotes the data block with index $j$.}
			\label{fig:merkle-tree}
		\end{figure}

		\begin{figure}
			\centering
				\begin{tikzpicture}[line cap=round, line join=round, yscale=0.8]
					% First layer
					\node [circle, draw=black, fill=white, minimum size = 1cm] (node-1-1) at (0, 0) {$H_0^r$};
					% Second layer
					\node [circle, draw=black, fill=white, minimum size=1cm] (node-2-1) at (-4, -2) {$H_1^r$};
					\node [circle, draw=black, fill=white, minimum size=1cm] (node-2-2) at (4, -2) {$H_2^s$};
					% Third layer
					\node [circle, draw=black, fill=white, minimum size=1cm] (node-3-1) at (-6, -4) {$H_3^r$};
					\node [circle, draw=black, fill=white, minimum size=1cm] (node-3-2) at (-2, -4) {$H_4^s$};
					% Fourth layer
					\node [circle, draw=black, fill=white, minimum size=1cm] (node-4-1) at (-7, -6) {$H_7^s$};
					\node [circle, draw=black, fill=white, minimum size=1cm] (node-4-2) at (-5, -6) {$H_8^r$};
					% Second data layer
					\node [draw=gray, text=gray, fill=white, minimum size=0.8cm] (data-2-2) at (-5, -8) {$D_4^s$};
					% First layer arrows
					\foreach \a in {1, 2}
					  \draw (node-1-1) -- (node-2-\a);
					% Second layer arrows
					\foreach \a in {1, 2}
					  \draw (node-2-1) -- (node-3-\a);
					% Third layer arrows
					\foreach \a in {1, 2}
					  \draw (node-3-1) -- (node-4-\a);
					\draw [draw=gray] (node-4-2) -- (data-2-2);
				\end{tikzpicture}
				\caption[Merkle Tree: Integrity Check]{An exemplary Merkle tree integrity check. A superscript $s$ denotes that the respective value is given by the untrusted sender, whereas a superscript $r$ denotes a value which is computed by the receiver.}
			\label{fig:merkle-tree-proof}
		\end{figure}

		A \emph{Merkle} tree, or \emph{hash tree}, is a tree in which every leaf node stores the hash of a data block and every non-leaf node stores the hash of the concatenated hashes of its children \cite{merkle1987}.
		This structure enables efficient, partial integrity checks on large data sets.
			
		\autoref{fig:merkle-tree} shows a binary Merkle tree with five leaves.
		As the node with index $4$ is a leaf, the stored hash $H_4$ can be computed as
		\begin{equation}
			H_4 = h(D_0),
		\end{equation}
		where $h(\cdot)$ is the employed hash function and $D_0$ is the corresponding data block for this particular leaf node.
		In contrast, the node with index $3$ is a non-leaf node and its hash $H_3$ can hence be calculated as
		\begin{equation}
			H_3 = h(H_7 + H_8),
		\end{equation}
		where $H_7 + H_8$ denotes the concatenation of the hashes of nodes $7$ and $8$.
		Similarly, the hash of the root node $H_0$, also referred to as \emph{root hash} or \emph{Merkle root}, results as
		\begin{align}
			H_0 &= h(H_1 + H_2)\\
			&= h(h(H_3 + H_4) + h(H_5 + H_6))\\
			&= h(h(h(H_7 + H_8) + h(D_0)) + h(h(D_1) + h(D_2)))\\
			&= h(h(h(h(D_3) + h(D_4)) + h(D_0)) + h(h(D_1) + h(D_2))).\label{eq:merkle-recursive}
		\end{align}

		\autoref{eq:merkle-recursive} indicates that, due to the recursive nature of the hash computation, there is no possibility of altering any data block without changing the root hash as well.
		Therefore, it is sufficient to know the root hash in order to be able to verify the integrity of any data block in the tree.

		For instance, a sender wants to send data block $D_4$ to a receiver, which only knows the root hash $H_0$.
		Let $D_4^s$ denote the sent data block, where the superscript $s$ indicates that this data comes from the untrusted sender and could be different from the original $D_4$.
		In addition to the possibly altered data block $D_4^s$, the sender sends the hashes $H_2^s$, $H_4^s$ and $H_7^s$.
		With the given data block and the given hashes, the sender then resconstructs the corresponding Merkle tree to compute the root hash $H_0^r$ (see \autoref{fig:merkle-tree-proof}), where the superscript $r$ indicates that a hash is computed by the receiver:
		\begin{align}
			H_8^r &= h(D_4^s)\\
			H_3^r &= h(H_7^s + H_8^r)\\
			H_1^r &= h(H_3^r + H_4^s)\\
			H_0^r &= h(H_1^r + H_2^s).
		\end{align}
		If the computed root hash $H_0^r$ equals the known root hash $H_0$, the sender has successfully verified the integrity of the given data block ($D_4^s = D_4$) without knowing any other data block or non-root hash beforehand.
		This holds since, with the use of a cryptographic hash function (see \autoref{subsec:hashing}), the sender cannot determine a hash $H_7^\prime$ which, combined with an altered data block $D_4^\prime$, would yield the correct parent hash $H_3$.\footnote{For this, the sender would have to solve $H_3 = h(H_7^\prime + D_4^\prime)$ for $H_7^\prime$, which requires the infeasible inversion of the cryptographic hash function $h(\cdot)$.
		One might argue that the hash $H_3^r$ does not need to equal $H_3$ as long as the resulting root hash $H_0^r$ equals $H_0$ --- but this just propagates the problem of solving for an unknown input to $h(\cdot)$ to another level in the tree.}
