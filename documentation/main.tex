\documentclass[a4paper]{article}

%% Language and font encodings
\usepackage[english]{babel}
\usepackage[utf8x]{inputenc}
\usepackage[T1]{fontenc}
\usepackage{float}


%% Sets page size and margins
\usepackage[a4paper,top=3cm,bottom=2cm,left=3cm,right=3cm,marginparwidth=1.75cm]{geometry}

%% Useful packages
\usepackage{amsmath}
\usepackage{graphicx}
\usepackage[colorinlistoftodos]{todonotes}
\usepackage[colorlinks=true, allcolors=blue]{hyperref}
\usepackage{listings}

\setcounter{tocdepth}{4}
\setcounter{secnumdepth}{4}

\title{Off-chaining Smart Contract Data to DBMS}
\author{Thanh Tuan Tenh Cong, 
Simon Fallnich,
Patrick Friedrich,
Tarek Higazi,\\
Vincent Jonany,
Dukagjin Ramosaj,
Kevin Marcel Styp-Rekowski}

\begin{document}
\maketitle

\begin{abstract}

Blockchain technology is increasing in significance. Many organizations in different industries, though primarily in the Financial industry, have begun to explore the advantages of this revolutionary technology and how it could benefit their businesses. The potential outcomes of the Blockchain are tremendous and it appears that any industry that uses some sort of transaction, are likely to be disrupted by the Blockchain technology. Even though Blockchain’s based applications benefiting from Blockchain’s most important properties of immutability and transparency, there are also costs to be considered. With today’s Blockchain implementations, there are a few issues that need to be addressed such as storage capacities and data computation costs, performance and scalability issues. Through this research project, we motivated the need for off-chaining approaches to overcome today’s Blockchain limitations and to answer our research question on: “How can data used on-chain be outsourced to a Relational Database Management System (RDBMS) while compromising blockchain properties as little as possible”. Hence, we built a prototype as a proof-of-concept and analyzed how well our approach performs in terms of solving this problem. We find that using off-chaining approaches makes sense for specific use cases. Therefore, we introduced two potentially relevant use cases and analyzed our off-chaining reasoning further in this report. Moreover, we evaluated our implementation through detailed benchmarking results which support further our arguments on when it makes sense to use our off-chaining approach.


\end{abstract}
\newpage
\tableofcontents
\newpage

\section{Introduction / Motivation}

One of the most significant technological achievements of this decade is the invention of Blockchain technology. In the span of 10 years this technology has sprouted a new concept which the world has come to know as crypto currencies, and which today hold a combined value of hundreds of billions of US dollars. \\

The Blockchain architecture gives us a unique combination of benefits which combine data integrity, transparency, security and peer-reviewed actions. 
“Blockchain-based applications, however, may also suffer from high computational and storage expenses, negatively impacting overall performance and scalability.” [ET 2017]\\

Along with the rising popularity of crypto currencies, and the following soar in their prices, one of the main obstacles which has been seen is the rise in execution time, due to the ever increasing amount of data being processed and the increasing amount of traffic, and this in turn fueled the rising values of the currencies even more.  \\

It is here where the problem we have focused on began to grow in significance. With the rising costs and execution times, the ability to store a lot of data on the Blockchain becomes less practical. \\

This creates a need for coming up with new approaches and ideas for storing large amounts of data in a Blockchain environment, while still benefiting from the features that come with it such as transparency, incorruptibility and data-integrity. This would enable an approach which would free developers from being limited or constrained by the execution times and costs associated with such large data storage, and consequently the same issues when it comes to editing and updating this data. The solution which we discuss and showcase in our paper is the approach of storing as much of the data as we can “off-chain” while preserving the integrity and the properties of the Blockchain.\\

“Off-chaining” is basically the act of moving data or computation flows outside of the Blockchain so that they can be stored or computed elsewhere. However storing data in different databases, servers or any other location comes at a cost, and the Blockchain’s core properties may not be possible to maintain. Ultimately, the framework should remain "trust-less", meaning that no definitive trust is required. Through this report we try to elaborate on our approach with minimal impact on compromising the blockchain properties.\\

In this paper we will describe how we developed a prototype which implements off-chaining with minimal cost to the Blockchain’s properties. We came up with different use cases where it would be practical to make use of Blockchain technology, and where the amount of data which needed storing would be too large to practically store on-chain.\\
We will show how are prototype was designed and how it functions. And we will lay out the results of our benchmarking and compare the costs and execution times of storing the data on-chain vs off-chain, and try to reach a conclusion on where and when this solution would be applicable and useful. 


\newpage
\section{Background}

\subsection{Blockchain}

A new concept for trustless, decentralized and transparent information processing and storage has been developed, introduced and already integrated into several mainstream applications in the last years. Bitcoin, developed by an unknown person or group under the name of Satoshi Nakamoto, led this initiative back in 2008 with its open-source, peer-to-peer network which cryptographically stores records in a chain and serves as a distributed ledger, this was named the Blockchain. The currency in which these operations were valued and paid for with is called Bitcoin, and it was the first decentralized digital currency.
Blockchains are essentially continuously growing lists of information which are linked together and secured using cryptography. The main benefit of Blockchains is that they are designed to be incorruptible, and that the data they hold cannot be unintentionally modified since the entire history of the chain is always stored and recorded on it. It is verified by peers and transparent to each user.
Blockchain technology is a mixture of various computing and economic concepts especially in financial sectors, including peer-to-peer systems, cryptography based, consensus protocols, decentralized storage, decentralized processing and smart contracts. The consolidation of these concepts makes blockchains as another innovation and as a programmable platform and system at the same time. Blockchains consist of specific properties including immutability and transparency of cryptographically-secured and peer-recorded transactions, which have been settled upon by consensus on the network. Being able to build up trustless interactions and business disintermediation continues to be one of the most important objectives of using blockchains.

\subsection{Smart Contracts}
Computer programs that are executed automatically when conditions (terms) of the contract have been met. Understanding smart contracts in a high-level aspect are the same as standard contracts. While a standard contract enforces the terms of a contract (regularly by law), a smart contract actualizes this enforcement of terms through network consensus and online cryptography secure systems. These programs are executed in a trustless and carefully designed way in the network and are referred as Smart Contracts. Ethereum is one of the most known platforms, it empowers developers to create their own smart contracts and creating decentralized applications in the blockchain [\cite{Buterin2014}]. Upon creation of smart contracts, they need to be executed in a secured runtime environment. Therefore, an Ethereum Foundation team project developed the Ethereum Virtual Machine (EVM) which is the runtime enviroment for Smart Contracts. Putting it in simpler understanding, EVM is a virtual machine that continues running on every node in the network and executes smart contracts. In addition, it is isolated that no framework or filesystem access is possible, thus, to ensure determinism. One vital part of the way the EVM works is that each and every task that is executed inside the EVM, is executed by each full node. This is a crucial part of the Ethereum 1.0 consensus model and has the advantage that any contract on the EVM can call some other contract at zero cost, however, it also has its disadvantages that the computational steps on the EVM are exceptionally costly [\cite{Buterin2014}]. In order that the Ethereum network is not being abused or being deliberately attacked, the Ethereum protocol charges a fee for every computational step. The way this cost is paid in Ethereum blockchain i’s through the attribute called gas. The fee or the price of the gas is deteremined on market-basis; meaning it is decided by the economic concept of market where supply meets demand among users and miners' willingness to mine the next block in the network (blockchain). Furthermore, understanding the way how fees work, is that every transaction alongside other attributes must contain the 'gasPrice', which is the fee that the transaction pays per unit of gas, and 'startGas' being the maximum amount of unit of gas that you are willing to pay for a transaction. Therefore, at every execution of transaction a prior evaluation of a transaction cost is done [\cite{Buterin2014}].
				\[ gasCost(Tx)= {gasPrice *  startGas}\]
To understand to which extent the usage of the EVM makes sense, it can mainly incorporate running business logic applications ("if this then that"), confirming signatures and other cryptographic objects [\cite{Ethereum2017}. On the other hand, any utilization which incorporates using EVM as a file storage, platform or anything to do with GUI, it does not fit under the reasonable use due to its tremendous costs. Therefore, usage of on-chain code execution with today's blockchain implementations, implies furthest to be at the stage of doing simple programming functions. Nevertheless, in order to be able to use Smart Contracts, higher level programming languages which compile EVM code have been developed such as Solidity, Serpent, LLL.



\begin{figure}[h]
\centering
\includegraphics[width=0.4\textwidth]{images/smartcontracts.png}
\caption{\label{fig:Smartcontracts}www.blockgeeks.com}
\end{figure}

\subsection{Integrity Checks}

Here goes my text.

\subsubsection{Hashing}
\subsubsection{Merkle Tree}
\subsubsection{Query Completeness}

\subsection{The Concept of Query Completeness} \label{sssec:querycompleteness}

The possibility to query RDBMS is a pivotal functionality of those systems and thus represents a great entry point for efforts to link the technologies of blockchain and RDBMS as targeted in this project. The trust that today’s users of RDBMS put into the correctness of the returned results for their queries could be nullified by the trustlessness offered by the blockchain - trustless query results in a sense. Right now, no user can be sure if her results were not tampered with or only part of the truth was returned to her.
Through an initial slide set from our supervisor Jacob Eberhardt we were introduced to the concept of query completeness which aims at achieving completely trustless queries. The general idea is to use the blockchain and its properties to counteract the four ways a database system could falsify the query results. In detail, the following measurements have to be prevented:
Firstly, the database system could try to not consider all database records while performing the query. This means, a mechanism has to be implemented that verifies that all records were looked at.
Secondly, the database system could try to add records to the query results that were not part of the database before. To counteract, a trustless system needs to show that all considered records were part of the database already and that the returned results are actual database entries.
Thirdly, the database system could try to leave out actual database records that fulfill the query in the returned set of records. Accordingly, we need to proof that that all records that fulfill a user’s query find their way into the results that the user obtains.
Finally, the database system could try to include actual database records that do not fulfill the query in the returned set of records. A trustless system thus has to check that only records that fulfill the query are returned to the user.
Any system that successfully counteracts these ways to counterfeit query results achieves query completeness as defined here. We will demonstrate our approach to implement query completeness in the chapter 4.3.3.3.

\section{Introduction / Motivation}

Here goes my text.
\section{Off-Chaining Data Implementation Strategy}

Here goes my text.


\section{Architecture}
\subsectionnames{Vincent Jonany}

This section aims to give a broad view of the components and how all of them work together to satisfy the needs of our prototype. 

\begin{figure}[t]%evtl:[t] [!htbp]
	\centering
	\includegraphics[width=1.0\textwidth]{images/architecture.png}
	\caption{\label{fig:architecture}The architecture of the off-chaining approach.}
\end{figure}

As seen in Figure \ref{fig:architecture}, we have divided our architecture into three sides:
\begin{itemize}
	\item Client side
	\item Database
	\item Blockchain environment
\end{itemize}

\subparagraph{Client Side}
The client side consists of the client side application, and the Ethereum node. The client side application is the biggest component as it bridges the database and the smart contract in the Ethereum node. Currently, we put trust into the client side to a certain extent. The extent of trust varies depending on the use case, but nevertheless, a certain level of trust has to exist. For example, upon insertion of new data, before any data goes into the smart contract to be processed, we trust that the client side application will not alter the values. This assumption also acts as a temporary solution to the problem which we have encountered in performing heavy computations inside the smart contract. This assumption has allowed us to take the computational burden from smart contract to the client side application. This problem and solution will be explained in more details in section \ref{subsec:approach-implementation-client}.

\subparagraph{Blockchain Environment}
The blockchain environment is the decentralized network where the smart contract and its local data are going to ultimately live in. The blockchain decentralized network is trustless. But what it truly means is that the trust is distributed amongst all nodes in the network. It also means that we do not enforce an external institution to make sure that the smart contract and its local data are accurate and consistent (integrity). The environment itself ensures the integrity of the data.

\subparagraph{Database}
The database however, is not trusted. Though in our approach we use the database to store data that is going to be used again in the smart contract, we cannot trust the database. The database is prone to internal attacks that can affect the integrity of the data. But at the same time, a database allows us to store large amount of data, and it can be easily integrated with other applications or systems, highly suitable for an off-chaining approach. Hence our approach includes leveraging data integrity check mechanisms when using off-chained data from an untrusted source, such as the database, in the smart contract.

\subparagraph{Transportation Layer}
We have not only made the assumption that our client side is trusted, but also that the transportation layer is secured. Prior to the assumption, we have thought of attacks such as the man-in-the-middle attack, and how detrimental this attack is when we want to save raw data to the database, or when communicating with the smart contract. For example, it could happen that the hashes created in the client side application are altered upon sending them to the smart contract during the initial step. The first step of the approach includes the client side application hashing the raw data and sending the hashes to the smart contract to be stored, mapping the off-chained data to their hashes. Hence, if an attack changes a hash to a different data’s hash, then someone may be able to pass the integrity check by reusing that altered hash’s raw data value.

\subparagraph{Application Flow}
There are two most basic and general approaches in which the user interacts with our client side application: inserting a new data, and performing a specific action with the off-chained data. In most general cases, the flow of our application when a user wants to insert a new data goes in the following way: 

\begin{enumerate}
	\item User posts a request to client side application with all the data that are required in the specified data model. 
	\item Client side application creates a Merkle tree using the data sent.
	\item Client side application sends the Merkle root hash to the the smart contract.
	\item Smart contract stores the root hash as a local variable.
	\item Smart contract fires an Event to the client side application to let it know if it has successfully stored it. 
	\item Client side application performs a database query to store the data and the root hash into the database.
	\item Database stores it.
	\item Return success message to the user. 
\end{enumerate}

In another general case, the flow of our application when a user wants to perform a specific task through the smart contract while using the off-chained data goes in the following way:

\begin{enumerate}
	\item User creates a request to the client side application to perform a specific smart contract function. 
	\item Client side application triggers relevant smart contract function.
	\item Smart contract triggers a smart contract event to the client side application to retrieve specific data from database. Specific data can be specified by using the root hash stored in smart contract previously.
	\item Client side application listens to that event and queries required data from the database.
	\item Client side application creates a Merkle tree from the queried data, and creates a Merkle proof.
	\item Client side application sends required data and proof to smart contract via function call.
	\item Smart contract performs integrity check using the proof from client side application.
	\item Smart contract continues with the original intended task requested by the user when the proof has passed the integrity check. 
	\item Smart contract computes a new root hash from the new changed value, and stores it.
	\item Smart contract triggers an event back to the client side application containing either the results, or an error when the proof does not satisfy the condition of the integrity check. 
	\item Client side application listens to event, and then stores the new root hash with the new changed value into the database. 
	\item Client side application finally returns either a success message, or a failure in case of a failed data integrity check. 
\end{enumerate}

\subsection{Technologies}

Here goes my text.

\subsubsection{NodeJS and Web3JS}
\subsubsection{Truffle, Smart Contracts (Solidity)}
\subsubsection{PostreSQL}
\section{Introduction / Motivation}

Here goes my text.


\subsubsection{Proof of Concept: Counters}

Here goes my text.
\subsubsection{Employee Use Case}

Here goes my text.

\paragraph{Concept}
\paragraph{Implementation}
\subsubsection{Financials Use Case}

\paragraph{Concept - Financials}
\subparagraph{Description}
An external auditor wants to check the financial situation of a company. One task could be to check the stated weekly or monthly sales amount against the sum of all sales records in the specific year. One of the first steps would be to verify (check the integrity) of all sales records in the specific year.

This use case consists of two parties, a financial auditor and the company that holds the financial data. An assumption that we make is that the middleware is open-sourced, produced by an auditing company, and it is being consumed by the company to store their financial data in the blockchain environment. The middleware allows the company to off-chain their financial records from the smart contract into a database, while maintaining the integrity of the data using the blockchain environment. The middleware also allows the auditor to easily audit companies’ financial records without having to worry about the integrity of data once they are appended. It also allows the auditor to query the financial records with a filter, such as the date of the records, performing query completeness. The middleware however does not allow users to edit the records through the middleware, though it is possible to do it on the database directly, which will then result in an integrity check error when the records are used or verified by the auditor.

\textit{Example: A Sales Table}
\begin{center}
    \begin{tabular}{| l | l | l | l | l |}
    \hline
    ID & Product & Date & Amount & Price \\ \hline
    1 & Wallet & 20171230 & 1 & 20 \\ \hline
    .. & ... & .... & .. & ... \\ \hline
    \end{tabular}
\end{center}

The different table rows represent moments in time (e.g. every Friday night after 00:00 or every first of the month) and thus new records are appended to the table. In this way, the records can be tracked back in time and an auditor could double-check the records for e.g. the last 6 months or the last 104 weeks. The smart contract entry then consists of the root hash over every table row (with each column being a leaf in the Merkle tree).

As the database is under the control of the company storing their financial figures, the financial auditor cannot be sure that those numbers were inserted correctly (same as without blockchain). By having the hashes of the records on the blockchain, the auditor can be sure that the company was not able to change the figures later on and thus has the certainty that there are no accounting tricks and malpractices in place e.g. at the end of the year or quarter. The figures that were once recorded by the company can be double-checked afterwards (or it can be noticed that the company recorded wrong numbers or tried to change them in hindsight). It is worth mentioning that the smart contract never stores the financial figures and they are thus not available on the blockchain either. While upon insertion, there isn’t a way to check that the numbers are true, but we can assert that these numbers will not be prone to illegitimate changes.

Furthermore, this use case could be valuable for rewarding benefits to employees. For example, the CEO of a company could receive a benefit by the shareholders (indirectly through the company itself but signed by the shareholders) if certain financial data are met. Or a sales person in the company could receive a benefit for sales data records for a certain month that went especially well.

To conclude, this use case asserts that large quantities of data (like financial figures of a company) cannot be modified after reporting them while only storing a relatively cheap representation of that data (hash) on the blockchain. Third parties and internal controllers are thus able to rely on the integrity of the recorded data.

\paragraph{Implementation - Financials}
\subparagraph{Initial Concept: Whole Table Verification}

In this initial approach and concept, we created an assumption that the financial rows are prone to changes, and that the smart contract is going to keep track of the total or the aggregation of the rows. Hence, a verification of the whole table is needed to ultimately maintain the integrity of the “total” row.

The smart contract stores the current root hash of the whole table. It could also store the root hashes of the prior states of the table (this information could otherwise be found in the blockchain history) to allow for additional functionalities.

The smart contract creates a Merkle tree over the hashes of all rows. The hashes of the rows are provided by the middleware. Afterwards, the smart contract verifies that the root hash of the created Merkle tree matches the current stored root hash. If that is the case, it calculates the new total value(s), and adds the hash of the new row (new entry) to the Tree to create the new root hash after these processes. This new root hash is then stored as the current one.

Merkle tree is used here when a column, or multiple column values are changed in a row, and we need to recalculate the new values for the "total" row. Instead of verifying the whole data in a row, we only need to verify the nodes that are affected by the change.

For the Merkle tree implementation, there is no need to implement a new one or change the existing one, as we can use the Multiple Item Proof here (note: an item here will correspond to a row, not a column).

Append row - Steps:
\begin{enumerate}
\item First, the client side application gets all the data from the Database.
\item It then creates the proof and sends it to the smart contract.
\item The smart contract does the integrity check.
\item Then the new total values are calculated.
\item The smart contract creates a root hash for that total row.
\item It then recreates the tree with that new roothash from the total row, and the hash of the newly added item row.
\item Store the new root hash of the whole table.
\item Send back the total row with its root hash, and the root hash of the added row.
\item Client side application receives it, and stores it into the database. 
\end{enumerate}

\subparagraph{Realization}

We quickly realized the “red flags” this approach is producing in regards to the gas cost. In the process of adding a new row, we are adding a new leaf, or a new hash into the current Merkle tree (step 6). This cannot work the way we imagined it, as we are planning on having only the Merkle proof, the nodes that are not affected by the change. Hence, to approach this we always have to send all of the leaves, the hashes of the rows, so that we can create this new tree in addition to a new leaf, the new hash of the inserted row. This quickly creates an overhead in the gas cost when adding a new row into the table. Hence, we took a turn in our approach, and we have to come up with something more usable, and worth it for the users. While the same gas cost problem will still exist in the next approach, it will however provide a much larger incentive for using the function. 

\subparagraph{Query Completeness}

The possibility to query RDBMS is a pivotal functionality of those systems and thus represents a great entry point for efforts to link the technologies of blockchain and RDBMS as targeted in this project. The trust that today’s users of RDBMS put into the correctness of the returned results for their queries could be nullified by the trustlessness offered by the blockchain - trustless query results in a sense. Right now, users may not be sure if the results were not tampered with or only part of the truth were returned to them.

% This is not needed (Vincent)
% Through an initial slide set from our supervisor Jacob Eberhardt we were introduced to the concept of query completeness which aims at achieving completely trustless queries. 

The general idea is to use the blockchain and its properties to counteract the four ways a database system could falsify the query results. In detail, the following measurements have to be prevented:
\begin{itemize}
\item Firstly, the database system could try to not consider all database records while performing the query. This means, a mechanism has to be implemented that verifies that all records were looked at.
\item Secondly, the database system could try to add records to the query results that were not part of the database before. To counteract, a trustless system needs to show that all considered records were part of the database already and that the returned results are actual database entries.
\item Thirdly, the database system could try to leave out actual database records that fulfill the query in the returned set of records. Accordingly, we need to proof that that all records that fulfill a user’s query find their way into the results that the user obtains.
\item Finally, the database system could try to include actual database records that do not fulfill the query in the returned set of records. A trustless system thus has to check that only records that fulfill the query are returned to the user.
\end{itemize} 

Any system that successfully counteracts these ways to counterfeit query results achieves query completeness as defined here. An example on how this mechanism is applied to our use case includes a user who wants to query all sales data in the period from Nov 2017 to Jan 2018, and expects that the results have not been tampered with.

\subparagraph{Initial Implementation Ideas}

Our initial approach to implementing query completeness includes, verifying that all entries were considered and that none were left out in the returned results. We can have the database responds to original query and to the negated query (e.g. original: all sales data in the period from Nov 2017 to Jan 2018, negated: all sales data NOT in the period from Nov 2017 to Jan 2018). And in the end, check if the combined size of both returned lists equals total number of entries (mapping counter in SC). Extending this idea, we can also send the two results, queried results, and the negated results to the smart contract to check if both lists will result into the stored root hash of the whole table.

Alternatively, we can verify that all results from the database match the query through smart contract, by storing all of the required data in the smart contract. 

% \item Verify that only actual entries from the database were returned
% → combine the two result lists to one Merkle proof to show that all rows are part of the table, i.e. no entries were changed and no outside entries were returned as a result

% \item Verify that the returned results match the query requirements (are “correct”)
% → no feasible way with blockchain currently as we would either need to do a lot of computation or to store a large amount of data (maybe even all data) on-chain, which would contradict this project’s focus
% → leave this verification up to the user (she can see whether the results match her query)
% \item Verify that all results that match the query were actually returned
% → no feasible way with blockchain currently as we would either need to do a lot of computation or to store a large amount of data (possibly even all data) on-chain, which would contradict this project’s focus
% → unfortunately, the user does not have an easy way to verify that either
% \end{itemize}

The above statements show that there is a clear trade-off between achieving query completeness and this project’s goal of saving gas costs by off-chaining data. Either practically all data has to be stored on-chain to enable our smart contract to guarantee query completeness, or a plethora of computations (Merkle proofs for each row in the data table) have to be on-chained which would hit the gas limit relatively fast.

As we are facing the challenge described previously, we decided to implement our proof-of-concept for query completeness by adding assumptions to make it feasible and easier for testing for large amount of data: building on our initial assumption that we trust our client side application, we calculate the Merkle proofs on the client side and thus save a considerable amount of computation on-chain. Nevertheless, the smart contract will still have to check the integrity of the data and perform a simplified query to guarantee query completeness. We also only allow the "date" column to be used in the query in this proof-of-concept.

\subparagraph{Proof of Concept}

This section describes the step-by-step actions between our client side application, smart contract and the database. The proof of concept includes the implementation of appending a new financial row, and query completeness.

\textit{Example: This table shows how the records are stored in the database, assuming all data here are appended through the smart contract.}
\begin{center}
    \begin{tabular}{| l | l | l | l | l | l | l | l |}
    \hline
    ID & company\_name & roothash & total\_sales & date & cogs & sc\_id & ... \\ \hline
    1 & CompanyAB & 0x123 & 1000 & 20160523 & 123 & 0 & .. \\ \hline
    2 & CompanyAB & 0x134 & 2000 & 20161215 & 421 & 1 & .. \\ \hline
    3 & CompanyAB & 0x321 & 3000 & 20170601 & 222 & 2 & .. \\ \hline
    4 & CompanyAB & 0x789 & 9000 & 20180130 & 980 & 3 & .. \\ \hline
    \end{tabular}
\end{center}



\textbf{\textit{Appending A Financial Row}}

\begin{figure}[h]%evtl:[t] [!htbp]
\centering
\includegraphics[width=1.0\textwidth]{images/appendRowFinancials.png}
\caption{\label{fig:appendRowFinancials}Activity diagram of appending a row to the financials record}
\end{figure}

\begin{enumerate}
\item User appends a new financial data, providing all the required information to be in the database
\item Client side application then creates a Merkle tree from the given raw data
\item Sends the root hash of the created Merkle tree to the smart contract
\item Smart contract appends that new root hash into its local mappings.
\item Smart contract increments the new length of the mapping, which is the smart contract ID that is going to be returned. This acts as an identifier to which hash belongs to which financial row. It is also used to preserve the ordering of the hashes which will later be used for query completeness.
\item Smart contract sends back the smart contract ID to the client side application
\item Client side application saves the raw financial data, with the root hash, and the smart contract ID in the database.
\end{enumerate}

\textbf{\textit{Query Completeness}}

\begin{figure}[h]%evtl:[t] [!htbp]
\centering
\includegraphics[width=1.0\textwidth]{images/queryCompleteness.png}
\caption{\label{fig:queryCompleteness}Activity diagram for query completeness}
\end{figure}


\begin{enumerate}
\item User requests to get all of financial data that are within the date range the user has requested.
\item The client side application gets all data from the database in the ascending order of the smart contract ID. This ID is generated by the smart contract to keep the ordering of the hashes that are stored in the smart contract. It is generated upon appending a financial record through the smart contract.
\item The client side application hashes each of the rows in the table.
\item It will then map the hashes of the row and the raw date value to the smart contract ID.
\item The client side application sends the mapping with the query to the smart contract.
\item The smart contract checks if it got the right data, the right amount of rows and the correct table overall by iterating over the provided root hashes and comparing them to the stored hashes in the local mapping.
	\begin{enumerate}
	\item If smart contract cannot verify the rows, then it will throw an integrity check error to the client side application as an event.
	\item If the smart contract can confirm this, it checks the query condition for every row and returns an array of booleans indicating all the indexes of rows that fulfill the query. Dates are intentionally stored as “uint” to ease the query function in the smart contract. For example 1, March, 2018 -> 20180301
	\end{enumerate}
\item Smart contract triggers an event to return back all the smart contract IDs that have satisfied the query.
\item The client side application listens to this event and returns the specified rows to the user subsequently.
\end{enumerate}

\subsection{Translator}

Here goes my text.
\section{Evaluation}

\begin{itemize}
\item See \href{https://drive.google.com/drive/folders/1KhEb6TT2YXKUlSJx2sdfR44ZhWUSHnXN}{Benchmarking strategy google sheet}
\item See what the results of those benchmarks are, make some nice figures out of them and put them in here
\item Evaluate / Conclude from those figures which findings were made
\end{itemize}

\subsection{Automated Benchmarking}

-- Describe automatization of benchmarking

-- Why was it needed?

Repeatable, fast, reusable

-- How was it done?

Accessing our application (middleware?) through the API-routes

Mention that application sends transaction (+ timeNeeded) with the JSON response to the client

Moccha, supertest (-> using test frameworks)

Writing to .csv file

.csv can then later be used to generate statistics

\subsection{Benchmarking for Employee Use Case}
(Maybe start text fancy with the questions which arise during the project: see the benchmarking google sheet for that)

As already mentioned this project was carried out as a prototyping project. Therefore we need to benchmark our results against the current best practice. Current state of the art for deploying contracts to the ethereum blockchain includes saving all the state variables of the contract on the blockchain and thus on all machines participating in the blockchain. As proposed we want to save those state variables off from the blockchain and thus save on gas which explains why gas cost are our most important measurement.

%- What did we benchmark?
%Explain which scenarios we did test (1 employee, multiple employee)
We especially benchmarked the employee use case as the employee use case is the more complex one of our both use cases. As a preparation for benchmarking we first needed two different smart contracts. One smart contract that works completely on-chain without using any external database, hashing functions or merkle trees and another smart contract that uses all introduced methods that are needed in order to off-chain our data to an external database. The two variants behave completely similar except for the saving of the state variables. Both variants are completely functional and could be used as they are if someone would like to deploy the employee use case. These fact were very important to us because only like that we could make sure that our benchmarks lead to meaningful results. Within the employee use case we measured different scenarios which will be described in the corresponding sections, mainly a time measure as well as a simple and another complex measure for the gas cost have been made.

%Explain under which aspects we did benchmark (gas, time)
%Mention again why gas is of utmost importance to us (because it translates to money and time)
%Explain the thing with benchmarking only the overhead due to automining of ganache
We mainly benchmarked the two measurements gas cost and time. Gas cost was the initial motivation for off-chaining as we assumed to save on gas cost when we off-chain the state variables. Gas cost translates to real currency as a participant of the blockchain either needs to buy Ether or help to mine blocks in order to receive Ether which is needed to pay the resulting gas cost of a transaction. The time was measured as we wanted to gain insights into how much overhead our application adds to a normal blockchain application. This was possible as our technology stack includes ganache which is capable of automatically mining new transactions into new blocks instantly so that we are able to measure only the overhead that our application introduces.

%- Nice transition into showing our first graphics.
\subsubsection{Time measurement}
First we will measure the introduced computation time of our application for executing different functions of the smart contract. The result can be seen in Figure \ref{fig:05_time}.

\begin{figure}[t]%evtl:[t] [!htbp]
\centering
\includegraphics[width=1.0\textwidth]{images/05_evaluation/05_time.png}
\caption{\label{fig:05_time}Time measurement for Employee Use Case.}
\end{figure}

%Thats a box plot chart
As we measured the computation time multiple times we decided to unite the results into a box plot chart. Thus we can see how the system behaves most of the times and single outliers can easily be identified. Throughout our measurement we always compare the on-chained approach against the off-chained approach. Every single function got called multiple times and the results have been summed up into the corresponding box plot.

For the \textit{employee contract} creation we can see the that the time differs by circa 50ms when comparing the on-chained approach to the off-chained approach. The on-chained approach needs about 400ms of computation time in the application while the off-chained approach needs 350ms. Both timings are negligible for us as the average block time of the ethereum blockchain is currently about 14 seconds [TODO: Need citation: https://etherscan.io/chart/blocktime]. Thus our application has more than enough time to compute a new transaction comparing nearly half a second against 14 seconds.

The \textit{payraise contract} creation needs about the same time for the on-chained as for the off-chained approach. Both need about 350ms of computation time and are negligible for the same reason as for the \textit{employee contract} creation. The same holds true for the \textit{add} functionality which takes even less time and thus the timing is negligible here as well.

More interesting are the timing needs of the \textit{increase salary} functionality. We can easily see that the off-chained variant has a much higher variance than the on-chained variant and we have peeks in the computation time of up to one second. Hereby it is very important to mention that the time needed for the off-chained variant increases with the size of the employee dataset. This happens because for every added employee the \textit{increase salary} function needs to generate another transaction. As every transaction needs to be mined first the increased timing needs of our application are also negligible as what we see in our chart is the cumulated computation time for all transactions that happened during this function call. All of these transactions need to be mined within blocks of the ethereum blockchain which takes a lot longer, namely about 14 seconds currently, than the computation of our application. Plus, it could possibly happen that the single transactions will be divided on multiple blocks. Therefore the introduced timing needs of our application are also negligible for this function and additionally all measurements of the \textit{increase salary} function were less than a second which is quite fast.% maybe explain that the introduced overhead is somehow "virtual?"

%Everything under 1 second -> not relevant, no further measurements
%Erwähnen dass es super ist, dass die middlewar keine extra zeiten introduced und genauso fix ist wie das normale.
As every function call in the off-chained approach does not introduce a real overhead compared to the on-chained approach and our application always performed under one second we concluded that we do not need to deeper analyze what introduces the most computation time or how to optimize for time in our application as this would not include a great benefit for our use case. Therefore no further measurements concerning the computation time have been made. We can conclude that the off-chained approach did not introduce any additional computation time and is as good as the on-chained approach timewise.

\subsubsection{Gas cost measurement}
Measuring the gas needed for executing the functionalities of our smart contract is of utmost importance to us. The initial motivation for our prototype was to gain experience on whether it is possible to save on gas cost when off-chaining state variables from smart contracts. As we receive the used gas for each transaction within the JSON-response when we call the smart contract functions through our API endpoint we were able to extract different insights into the use of gas when on- or off-chaining.

%And smth more in Figure \ref{fig:05_gas_cost_single}.
In Figure \ref{fig:05_gas_cost_single} we visualized the used gas for calling single functions of our smart contract for the employee use case. Similarly to our time measurement we again measured the main functionalities of our smart contract and compared the on-chained variant with the off-chained variant.

\begin{figure}[t]
\centering
\includegraphics[width=1.0\textwidth]{images/05_evaluation/05_gas_cost_single.png}
\caption{\label{fig:05_gas_cost_single}This is one employee.}
\end{figure}

%- employee creation more costy because overhead of functions for merkle tree (elaborate what exactly here)
% 844906 - 714518 = 130388 gas,    130388 / 714518 = 18,25%
%    results) in a future work which could be mmaking library identifies future work
% abgesehen von leeren Listen 
%Say why we save and then in the end say that its constant and how much percentage-wise this translates to
%Blabla could be optimized by making those functions libraries
For the employee contract creation we can see that the off-chained variant needs about 130 thousand more gas than the on-chained variant. Both Smart Contracts do not save any state variables yet apart from empty lists for either hash values or employees, depending on the approach. The difference in the gas cost originates mainly from the added functions for the merkle tree proof and verification which were added to the smart contract for the off-chained approach. Thanks to this finding a new task for future work could be identified. Extracting the functions which are needed in order to iterate through the tree and verify the merkle proof into a library in the sense of Solidity would benefit the off-chaining approach. Calling functions of libraries in Solidity is in general very beneficial in terms of gas cost (TODO: better word than 'beneficial': Isn't it free?). Anyways the gas cost of the off-chained approach exceed the gas cost of the on-chained approach only by approximately 18\% which is rather minor. All in all, it can be said that the increased gas cost of the off-chained approach are negligible as the contract creation needs to be paid only once throughout the whole life cycle of the smart contract. Analyzing the offered functions of the smart contract is much more important as these will be called multiple times.

TODO: Check that this is really mentioned in our future work part.

- payraise not changed (on-chained for both, thats why it is the same)

- on add we save xy gas because thats where off-chaining is stronk

- increase salary we loose because [.....] only one field, multiple transactions, elaborate on possible alternatives like functions that would change multiple strings etc etc

And yet smth more in Figure \ref{fig:05_gas_cost_ten}.

\begin{figure}[htbp]
\centering
\includegraphics[width=1.0\textwidth]{images/05_evaluation/05_gas_cost_ten.png}
\caption{\label{fig:05_gas_cost_ten}This is ten employees.}
\end{figure}



- Insert graphics / statistics

Explain what is visible

Explain what that does mean

- Elaborate on conclusions and findings from this evaluation

What do all the statistics say? what do they mean to us?

When does Off-Chaining make sense?

Do we really save gas?

- Later on say how our measurements can be compared to the financials use case (cut out the increase-salary basically)

- Know your use case!

- Give pay-raise as example for when off-chaining makes no sense (we kept it on-chain)

-> Conclude on that

\section{How to Deploy / How to use / User Guide}



\begin{itemize}
\item Describe the whole process which would be needed in order to change a on-chain contract to its off-chained counterpart.
\item Use Library
\item Use Events, routes
\item Listen for events, add database models
\end{itemize}

\subsection{Describe complete building process}


\newpage

\section{Future Work}

% \begin{itemize}
% \item Essentially, copy our future work parts from the presentations :)
% \end{itemize}
In this section, we aim to provide the possible future steps that can show and increase the value or the advantages of the off-chaining approach. In the course of our project, we have identified other ways that we could have done to accomplish this. We have also identified other potential integrity check mechanisms that can perform more efficiently, such as paying less gas cost. 

\subsection{Make Middleware Trustless}

\subsection{Oraclize Approach to Save Gas}
During our research about alternative approaches to check the integrity of the off-chained data, we analyzed the Oracle provider Oraclize (see part 3.2) and found an approach that Oraclize uses to save gas costs which could present a further opportunity to extend this project’s system as well. By giving its users the choice whether to provide the proof that the data pushed by their Oracle is correct to the Smart Contract that requested the data in the first place and that may want to run the integrity check before actually using it or to store the proof for later use on a trustless database (IPFS) Oraclize can save its users a lot of gas in contrast to running the proof of correct data on-chain every time data is pushed to a Smart Contract. Accordingly, the presented system in this paper could provide its users with the same choice and further improve on its goal to save gas. At the same time, the integrity of the data can still be secured.

\subsection{Merkle DAG}
\subparagraph{Concept}
Merkle DAG (Directed Acyclic Graph) is one of the features that we included into our potential roadmap after the mid-term presentation. However, due to the limited time and resources we have, we decided to prioritize other things in the roadmap.

The data structure compatible in the application side will affect the complexity of the use cases that we can handle. However, at the same time, we also try to not tremendously increase the gas cost. The Merkle DAG increases our complexity of the data structure from a binary tree to a directed acyclic graph. It allows us to have a more complex relationship. For example, we could create a complex Role Based Access management application that leverages the Smart Contract to maintain the integrity of the hierarchy status of each role. 

We could of course implement other data structure that allows more complex behavior, however as mentioned previously, we want to keep the gas cost considerable. Hence, the second requirement requirement is that the data structure has to have an efficient integrity check mechanism. As the name suggests, Merkle DAG also implies the similar Merkle proofing mechanism that a Merkle tree has. Hence, an integrity check on the root hash, to make sure that the children have not been changed is efficient as well. 

Such complexity and a level of efficiency in the data integrity check this mechanism provides can provide higher values to the users. Hence, incentivizing them to use the off-chaining approach in comparison to the traditional approach. 

\subparagraph{Related Works}
Over the past few years, some papers have been published in regards to extending the Merkle hash technique from just trees but to other data structures, such as directed acyclic graphs. <http://truthsayer.cs.ucdavis.edu/model.12.6.pdf> In addition to that, another paper has been published in regards to a revised hashing technique for directed acyclic graphs <https://eprint.iacr.org/2012/352.pdf>. The paper suggests that the traditional Merkle hashing technique is not suitable for a more complex data structure, such as the directed acyclic graph. And lastly, the closest work to a proof of concept for Merkle DAG <https://github.com/jbenet/random-ideas/issues/20>.  //TODO Cite stuff

\subsection{Possible another integrity check mechanisms}


\section{Project Organization}

This chapter aims to provide an overview on our internal project organization. Firstly, we are going to cover our software development cycle. We will then provide brief descriptions on the methods and technologies we used to help us maintain the best practices in our software development cycle. 

We chose to follow an agile software development methodology and we used SCRUM as our framework. We mainly chose SCRUM due to the nature of our project - it is a highly complex and potentially huge project, and we needed constant guidance from our supervisor. SCRUM encourages short, iterative sprints, weekly meetings, and daily stand-ups. In addition to that, the number of members in the group seems very appropriate for SCRUM. Hence, we made SCRUM the backbone of our project organization. 

As we have been given a fixed schedule for the project, we were able to create two iterative cycles, each cycle ending with a presentation of our current findings and results to everyone who takes part in the project. The cycle consists of brainstorming, research, implementation, and a demo. 

In addition to our quarterly presentations, every two weeks we demo the current prototype to our product owners, which are the project supervisors. Every week the team members, excluding the previous roles mentioned previously, meet two times a week for a general discussion and sprint planning respectively. We chose to have shorter sprints (weekly), because we learned that it was very easy for us to lose track from the project goal. Hence, the weekly meeting will help us to stay on track. Moreover, we also have daily virtual stand-up meetings. This aims to further prevent one another from going out of the scope, and for everyone to be more aware of what each member is currently doing, or if they are blocked with the current task, and require immediate reinforcements. 

We use several technologies and mechanisms to aid us in following SCRUM with the best practices. One of the most important tool we use is ClickUp. ClickUp is a project management platform that helps us manage and track tasks, stories, epics, and sprints. During our sprint planning, our project manager would create all the necessary elements for the developers to start with their tasks. For example, creating a new sprint board, with all the members' tasks. The developers must then proceed with adding their own descriptions, user stories, and acceptance criteria into their own tasks. At the same time, the project manager can move and add tasks into and out from the backlog, depending on the resource allocation of each member for that sprint. 

As the sprint starts, ClickUp still plays a huge role during the development phase. Another tool that we strongly leverage on is GitHub, a version control for collaboration platform. Since each task is isolated, ClickUp allows users to track each commit and/or branch to each task, making our pull-requests organized, understandable and also isolated.

As a team, we also created a guideline on how to better isolate and organize branches, commits and ultimately the tasks themselves. For every task created on ClickUp, there is a unique identification which is used by us to link the name of the branches and commits. For example,

\begin{itemize}
\item For branches: \_\#\{task id\}\_short\_description\_of\_branch.
\item For commits: \#\{task id\}\_commits\_message.
\end{itemize}


Once developers are done with their features, they create a pull request and they can move that specific task to the "QA" category in ClickUp for another developer to review the pull request. We make sure that we have at least two approvals before the pull request can be merged, and only that assigned reviewer can move the task to "Done".

We heavily rely on Slacks and Google Hangouts for our day-to-day communication between team members, including the product owners, and our daily stand-up calls or messages. 

Lastly, we use Google Drive to store our presentations, research documents, schedules, and most importantly, the meeting protocols. These protocols are created whenever there are more than three people meet together for the project. The protocol should contain all the members present during that meeting, the agenda of the meeting, discussions, and next steps. The protocol helps project members to look back on what have been discussed in regards to design decisions and current of future tasks. 


\section{Conclusion}

Here goes my text.

\newpage
\bibliographystyle{unsrt}
\bibliography{sample}

\end{document}