\section{Background}

\subsection{Blockchain}

A new concept for trustless, decentralized and transparent information processing and storage has been developed, introduced and already integrated into several mainstream applications in the last years. Bitcoin, developed by an unknown person or group under the name of Satoshi Nakamoto\cite{nakamoto2008bitcoin}, led this initiative back in 2008 with its open-source, peer-to-peer network which cryptographically stores records in a chain and serves as a distributed ledger, this was named the Blockchain. The currency in which these operations were valued and paid for with is called Bitcoin, and it was the first decentralized digital currency.
Blockchains are essentially continuously growing lists of information which are linked together and secured using cryptography. The main benefit of Blockchains is that they are designed to be incorruptible, and that the data they hold cannot be unintentionally modified without mutual consent since the entire history of the chain is always stored and recorded on it. It is verified by peers and transparent to each user.
Blockchain technology is a mixture of various computing and economic concepts especially in financial sectors, including peer-to-peer systems, cryptography based, consensus protocols, decentralized storage, decentralized processing and smart contracts. The consolidation of these concepts makes blockchains as another innovation and as a programmable platform and system at the same time. Blockchains consist of specific properties including immutability and transparency of cryptographically-secured and peer-recorded transactions, which have been settled upon by consensus on the network. Being able to build up trustless interactions and business disintermediation continues to be one of the most important objectives of using blockchains.

\subsection{Ethereum and Smart Contracts}
Smart contracts are a feature which the Ethereum Blockchain is based on. A smart contract is basically a computer program which can check for the fulfillment of certain preconfigured conditions and control the transfer of funds or other assets between several parties. While a standard contract enforces its terms through its legality, a Smart Contract actualizes this enforcement automatically through network consensus and a cryptographically secured system. The contracts are stored on the Blockchain and thereby serve as a decentralized middle man which stores records, enforces conditions and is completely neutral and transparent. The most popular platform which utilizes this technology is the Ethereum Blockchain. It empowers developers to create their own Smart Contracts and their own decentralized Blockchain applications [\cite{relatedWork38}].  \\

The runtime environment which the Smart Contracts actually run on is called the Ethereum Virtual Machine (EVM), and it continuously runs on every node on the chain, which means that every task which is executed inside of the EVM is run by each node, which is obviously costly, but one of the benefits of this is that each contract can call any other contract at zero cost [\cite{relatedWork38}]. The EVM is also isolated in the sense that no outside framework or file system is accessible, this to ensure determinism. The Smart Contracts themselves are written using the Solidity programming language which was designed for this purpose, and it allows users to create contracts which can be used for voting, crowdfunding, blind auctions, multi-signature wallets and more.\\

In order to ensure that the Ethereum network would not be abused or deliberately attacked, the Ethereum protocol charges a fee for every computational step. The way this cost is paid is through an attribute called ‘gas’. The fee or the price of the gas is determined simply via supply and demand; the users’ willingness to pay vs the price for which the miners' are willing to mine the next block in the Blockchain. Basically the way these fees work is that every transaction contains a 'gasPrice' attribute, which is the price per computational step, and the 'startGas' attribute, being the maximum amount of gas which the sender is willing to pay for the transaction. Therefore, at every execution of a transaction a prior evaluation of the transaction cost is done [\cite{relatedWork38}].\\

				\[ gasCost(Tx)= {gasPrice * startGas}\]
\\
\\
The usage of the EVM for the most part makes sense when it comes to running business logic applications ("if this, then that") such as confirming signatures or other cryptographic objects [\cite{relatedWork39}. On the other hand, any utilization which incorporates using the EVM as a file storage platform or anything GUI related would not be practical due to the tremendous costs.



\begin{figure}[h]
\centering
\includegraphics[width=0.4\textwidth]{images/smartcontracts.png}
\caption{\label{fig:Smartcontracts}www.blockgeeks.com}
\end{figure}

\subsection{Integrity Checks}

Here goes my text.

\subsubsection{Hashing}
\subsubsection{Merkle Tree}
\subsubsection{Query Completeness}


\newpage
